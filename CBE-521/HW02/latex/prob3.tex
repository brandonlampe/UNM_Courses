\documentclass[10pt, reqno]{amsart}
\usepackage[margin=1in]{geometry} % see geometry.pdf on how to lay out the page. There's lots.
\geometry{letterpaper}
\newtheorem{mydef}{Statement}

\title{CBE 521\\ Advanced Transport Phenomena\\ Homework 2}
\author{Brandon Lampe}
\date{Sept. 29, 2015} % delete this line to display the current date

%%% BEGIN DOCUMENT
\begin{document}

\maketitle
\section{Problem 3: Steady Parallel Rectilinear Flow}
\subsection{Part b: Show the solution can be written as the sum of particular and complementary (homogeneous) solutions, where $v=v_p + v_c$}

\begin{mydef}
The complete solution is:
% problem statement
\begin{equation} \label{eq:PDE}
\frac{\partial^2 v}{\partial \xi^2}+\frac{\partial^2 v}{\partial \eta^2}=-1
\end{equation}
\end{mydef}

the problem domain:
\begin{gather} \label{eq:bcts}
 v(\pm1,\eta)=0\\
 v(\xi , \pm r)=0
 \end{gather}

\begin{mydef}
Must make sense of problem domain definitions:
\begin{gather}
 x \in [0, 2B] \\
 y\in [0,2W] \\
 r \equiv \frac{W}{B} \\
\xi \equiv \frac{x}{B} =\frac{2B}{B} = 2 \implies -1\le \xi \le 1\\
\eta \equiv \frac{y}{B} = \frac{2W}{B} = \frac{2rB}{B} = 2r \implies -r \le \eta \le r
\end{gather}
\end{mydef}

\begin{mydef}
Separate the differential equation into particular and homogeneous parts, and let the particular solution $(v_p)$ satisfy the boundary conditions given in equation \ref{eq:bcts}:
\begin{gather} \label{eq:part}
\frac{\partial^2 v_p}{\partial \xi^2} = -1\\
v_p(\pm 1, \eta) = 0
\end{gather}
\end{mydef}

\begin{mydef} The complementary part of the problem is then given by:
\begin{gather} \label{eq:hom}
\frac{\partial^2 v_c}{\partial \xi^2}+\frac{\partial^2 v_c}{\partial \eta^2}=0\\
% v_c(\pm 1, \eta)=0\\
% v_c (\xi, \pm r)=-v_p(\xi, \pm r)
\end{gather}


The complementary solution must satisfy the following:
\begin{equation}
v - v_p = v_c \implies \frac{\partial^2 v}{\partial \xi^2}+\frac{\partial^2 v}{\partial \eta^2} - \frac{\partial^2 v_p}{\partial \xi^2} = -1 - (-1) = 0
\end{equation}

The boundary conditions must then be:
\begin{gather}
 v(\pm1,\eta) - v_p(\pm1, \eta) = v_c(\pm1, \eta)=0 - 0 = 0\\
 v(\xi,\pm r) - v_p(\xi, \pm r) = v_c(\xi, \pm r) \implies 0 - c_1 = c_1
\end{gather}
Therefore:
\begin{equation}
c_1 = v_c(\xi, \pm r) = - v_p(\xi, \pm r)
\end{equation}
\end{mydef}

\subsection{Part c: Show the particular solution:}
\begin{mydef}
The solution to equation \ref{eq:part} is obtained by integrating twice and solving for the boundary conditions:
\begin{gather}
v_p(\xi = \pm 1, \eta) = -\frac{\xi^2}{2} + \frac{1}{2} = \frac{1}{2}(1-\xi^2)
\end{gather}
\end{mydef}

\subsection{Part d: Obtain the complementary solution:}
\begin{mydef}
Considering only the homogeneous form of the equation will allow for a system of basis functions that satisfy the given boundary conditions, and a solution method similar to that for the Laplace equation will be utilized.  Additionally, the boundary conditions of the homogeneous form will be modified such that the boundary conditions for $v$ are satisfied (i.e., $v_c(\xi, \pm r) = -v_p(\xi, \pm r)$). The Homogeneous form of the equation is:

\begin{equation} \label{eq:HomPDE}
\frac{\partial^2 v_c}{\partial \xi^2}+\frac{\partial^2 v_c}{\partial \eta^2}=0
\end{equation}

$$v_c(\pm 1, \eta)=0$$
$$v_c (\xi, \pm r)=-v_p(\xi, \pm r)=\frac{\xi^2}{2} - \frac{1}{2}$$
\end{mydef}

\begin{mydef}
Seperate the variables and assume $v_c\left(\xi, \eta\right)$ consists of two independent functions $\Xi\left(\xi\right)$ and $H \left(\eta\right)$: 
$$v_c \left( \xi,\eta \right) = \Xi \left( \xi \right) H \left( \eta \right)$$ 
\end{mydef}

therefore, using the product rule, the partial derivatives are:
$$v_{c,\xi \xi} = \Xi^{''} H $$
$$v_{c,\eta \eta} = H^{''} \Xi$$

\begin{mydef}
Substitute the above definitions into equation  \ref{eq:HomPDE} and the following (equation \ref{eq:sep}) is obtained. If the left and right sides of the above equation are equal for all $\xi$ and $\eta$, then $\lambda$ must be a constant.
\begin{equation} \label{eq:sep}
	\Xi^{''}(\xi) H(\eta)=-H^{''}(\eta) \Xi(\xi)=\lambda = \frac{\Xi^{''}(\xi)}{\Xi(\xi)}=-\frac{H^{''}(\eta)}{H(\eta)}
\end{equation}
\end{mydef}

\begin{mydef}
Because $\lambda$ is constant, equation \ref{eq:sep} may be rewritten in terms of two ordinary differential equations.  
\begin{equation} \label{eq:ode1}
\Xi^{''} - \Xi \lambda^2 = 0 
\end{equation}
\begin{equation} \label{eq:ode2}
H^{''}+H\lambda^2 = 0
\end{equation}
\end{mydef}

\begin{mydef}
By translating the boundary conditions to be in terms of the separation variables and dividing them to be in terms of a single variable results in the following:
\begin{gather*}
	v_c(+1,\eta) = \Xi(+1)H(\eta) = 0 \hspace{0.25in} \forall \, \eta \in[-r,+r] \implies \Xi(+1)=0 \\
	v_c(-1,\eta) = \Xi(-1)H(\eta) = 0 \hspace{0.25in} \forall \, \eta \in[-r,+r] \implies \Xi(-1)=0 \\
	v_c(\xi,-r) = \Xi(\xi)H(-r) = 0 \hspace{0.25in} \forall \, \xi \in[-1,+1] \implies H(-r)=0 \\
	v_c(\xi,+r) = \Xi(\xi)H(+r) = 0 \hspace{0.25in} \forall \, \xi \in[-1,+1] \implies H(+r)=0
\end{gather*}
\end{mydef}

\begin{mydef}
The equations (Eq. \ref{eq:ode2}) is now an ordinary differential equations with homogeneous boundary conditions having the  characteristic equation $m^2 + \lambda^2=0$. The solution of this system is in form of an infinite sequence of eigenfunctions $(H_n)$ and eigenvalues $(\lambda_n)$.
\begin{gather}
H^{''} + H\lambda^2 = 0 \\
\nonumber H(+r)=0\\
\nonumber H(-r)=0
\end{gather}
\end{mydef}

\begin{mydef}
The solution to \ref{eq:ode2} can be written in the form: $H=e^{\alpha \eta} \left( Asin(\beta \eta) + Bcos(\beta\eta) \right)$. Here $\alpha=0$ and $\beta=\lambda$. Substitution of $(\eta + r)$ for $\eta$ in the equation result only in a shift in the eigenspace and allows for a solution when the boundary conditions are applied:
\begin{gather*}
H=Asin(\beta (\eta + r)) + Bcos(\beta((\eta+r))\\
H(\eta = -r)=0\\
H(\eta = +r)=0
\end{gather*}
\end{mydef}

\begin{mydef}
Solving for the boundary conditions at $\eta = -r$ results in $B=0$. This leaves $H=Asin(\lambda(\eta + r))$, and when solved at the boundary $\eta = +r$:
\begin{gather}
H(+r)=0=Asin(\lambda2r)
\end{gather}
If $A=0$, the solution reduces to the trivial solution $H=0$. For a non-trivial solution $sin(\lambda(2r))=0$ and by letting $\lambda(2r)=(n+1)\pi, \, n=0,1,2,...$ a solution for $\lambda$ may be obtained:
\begin{gather}
\lambda_n=\frac{(n+1)\pi}{2r}, \, n=0,1,2,3,...\\
H_n(\eta)=sin\left(\frac{(n+1)\pi(\eta+r)}{2r}\right), \, n=0,1,2,3...
\end{gather}
\end{mydef}

\begin{mydef}
Solve the remaining ODE (Equation \ref{eq:ode1}) with boundary conditions:
\begin{gather}
\Xi^{''}-\Xi\lambda^2=0\\
\nonumber \Xi(\xi=+1)=0\\
\nonumber \Xi(\xi=-1)=0
\end{gather}
\end{mydef}

\begin{mydef}
The remaining equation has the characteristic equation of $m^2-\lambda^2=0 $.  By performing similar steps to those shown in statements 11 and 12 above, it may be shown that $\Xi = Asinh(\lambda(\xi+1))$ and for a nontrivial solution:
\begin{gather}
\lambda_m = \frac{(m+1)\pi{\xi+1}}{2}, \, m=0,1,2,3...\\
\Xi_m(\xi) = sinh\left( \frac{(m+1)\pi(\xi+1)}{2} \right), \, m = 0,1,2,3...
\end{gather}
\end{mydef}

\begin{mydef}
the complementary solution is then obtained when the partial second derivatives of $H_n$ and $\Xi_m$ are calculated and the solution is summed over $m$ and $n$.  This results in:

\begin{gather*}
v_c \left( \xi,\eta \right) = \Xi \left( \xi \right) H \left( \eta \right)=
\frac{\partial^2 v_c}{\partial \xi^2}+\frac{\partial^2 v_c}{\partial \eta^2} =\\ 
-\left( \left(\frac{(m+1)\pi(1+1)}{2}\right)^2 \left( \frac{(n+1)\pi(1+r)}{2r} \right)^2 \right)
\sum\limits_{m=0}^\infty \sum\limits_{n=0}^\infty
sinh\left( \frac{(m+1)\pi(\xi+1)}{2} \right)
sin\left(\frac{(n+1)\pi(\eta+r)}{2r}\right)
\end{gather*}
\end{mydef}

\subsection{Part e: Show that as the aspect ratio $r$ grows large, teh velocity distribution approaches that of the particular soluation $v_p$}

\begin{mydef}
Because the aspect ration $r$ is in the denominator of $v_c$, as $r \rightarrow \infty$, $v_c \rightarrow 0$.  This results in:

\begin{equation}
r \rightarrow \infty: v=v_p + v_c = v_p + 0 = v_p
\end{equation}

\end{mydef}

\end{document}