% --------------------------------------------------------------
% This is all preamble stuff that you don't have to worry about.
% Head down to where it says "Start here"
% --------------------------------------------------------------
 
\documentclass[10pt, letterpaper]{article}
 
\usepackage[margin=1in]{geometry} 
\usepackage{mathtools, amsthm, amssymb, changepage, enumitem}
\usepackage[english]{babel}
\usepackage{bm}
\usepackage{tensor}
\usepackage{xfrac}
\usepackage{gensymb}

\renewcommand\thesection{ \arabic{section}}
\renewcommand\thesubsection{(\alph{subsection})}
\renewcommand\thesubsubsection{(\roman{subsubsection})}
 
\begin{document}
 
% --------------------------------------------------------------
%                         Start here
% --------------------------------------------------------------
 
\title{ASSIGNMENT 4}%replace X with the appropriate number
\author{Brandon Lampe\\ %replace with your name
ME 512 - Continuum Mechanics} %if necessary, replace with your course title
 
\maketitle

%=====================================
% PROBLEM 1
%===================================== 
For the basis $\bm{e_i}$, the components of $\bm{T}$ are:  
		$\begin{bmatrix} 3 & 0 &-1 \\ 0 & 1 & 0 \\ -1 & 0 & 3\\
						 \end{bmatrix} $
\section{Find the components of $\bm{T^2}$ and $\bm{T^3}$ for the $\bm{e_i}$ basis:}
	\begin{itemize}
		\item $\bm{T^2} \Rightarrow \overset{e-e}{[T]} \overset{e-e}{[T]} = \begin{bmatrix}
			10 & 0 & -6 \\
			0 & 1 & 0 \\
			-6 & 0 & 10
			\end{bmatrix}$

		\item $\bm{T^3} \Rightarrow \overset{e-e}{[T]}\overset{e-e}{[T]}\overset{e-e}{[T]}= \begin{bmatrix}
			36 & 0 & -28 \\
			0 & 1 & 0 \\
			-28 & 0 & 36
			\end{bmatrix}$	
	\end{itemize}
% ==== problem 2 ====
\section{Find $I_T = tr(\bm{T}), II_T = tr(\bm{T^2}), III_T = tr(\bm{T^3})$:}
	\begin{itemize}
		\item $I_T = 3+1+3 = 7$
		\item$II_T = 10 + 1 + 10 = 21$
		\item$III_T = 36 + 1 + 36 = 73 $
	\end{itemize} 
	
% ==== problem 3 ====
\section{Find the eigenvalues of $\bm{T}$ and the eigenvectors.  Construct a principal basis ($\bm{p_a}$) expressed in terms of 	$\bm{e_i}$}

	\begin{itemize}
		\item The eigen problem:  $\bm{T} \cdot \bm{p} = \bm{\lambda p} \Rightarrow
				\overset{e-e}{[T]} \overset{e}{\{p \}} = \lambda \overset{e}{\{p\}} \qquad	
				or \qquad \Big[ \overset{e-e}{ [T]} - \lambda [I] \Big] \overset{e}{\{p \}}  = \{0\}$
			\begin{itemize} 
				\item where: \\ $\lambda = $ eigenvalue \\ $\{p\} = $ eigenvector
				\item a non trivial solution (non zero) for $\overset{e}{\{p\}}$ in 
					$\Big[ \overset{e-e}{ [T]} - \lambda [I] \Big] \overset{e}{\{p \}}  = \{0\}$ only exists if the determinant:
					$det \left( \Big[ [T] - \lambda [I] \Big] \right) = 0$.  The real values of $\lambda$ that satisfy this 
					 are the eigenvalues of $[T]$.
%				\item this equation can be solved via Cramer's rule, where:
%					 \begin{align} %\begin{equation*}
%						\{ p \} & = \Big[ [T] - \lambda [I] \Big]^{-1} \{0\} \\
%							\Big[ [T] - \lambda [I] \Big]  ^{-1} & = \frac{\Big[ [T] - \lambda [I] \Big]^{cf}}
%							{det \left( \Big[ [T] - \lambda [I] \Big] \right)}
%					\end{align} %\end{equation*} 
%				\item however, this method would result in a very tedious process; therefore, an alternate method
%					 will be used (shown below).			
			\end{itemize}
		\item the equation: $det \left( \Big[ [T] - \lambda [I] \Big] \right) = 0$; is titled the characteristic equation of 
			$\Big[ [T] - \lambda [I] \Big] $
		\item a characteristic equation can be shown in terms of a polynomial function (via Leibniz' rule), therefore:
			\begin{itemize}
				\item characteristic equation:  $det \left( \Big[ [T] - \lambda [I] \Big] \right) = 0$
				\item characteristic polynomial:  $\lambda ^3 - I^* \lambda^2 - II^* \lambda - III^* =- P(\lambda) = 0$
			\end{itemize}
		\item find the characteristic invariants of $[T]$:
			\begin{itemize}
			\item characteristic equation:  $det \left( \Big[ [T] - \lambda [I] \Big] \right) =
				\lambda ^3 - I^* \lambda^2 - II^* \lambda - III^* =- P(\lambda) = 0$ \\
				where a ${}^*$ indicates a \textbf{characteristic} invariant
			\item $I^*_T  = I_T = tr\left( \overset{e-e}{[T]} \right) = 7$
			\item $II^*_T = \frac{1}{2}(II_T - I^2_T) = det \begin{bmatrix}
				T_{11} & T_{12} \\ T_{21} & T_{22} \end{bmatrix} + det \begin{bmatrix}
				T_{11} & T_{13} \\ T_{31} & T_{33} \end{bmatrix} + det \begin{bmatrix}
				T_{22} & T_{23} \\ T_{32} & T_{33} \end{bmatrix} = -14$
			\item $III^*_T = \frac{1}{6}(I^3_T - 3I_T II_T + 2III_T) = det\left(\overset{e-e}{[T]} \right) = 8$
			\end{itemize}
		\item input invariants into the characteristic equation for $\overset{e-e}{[T]} \Rightarrow
				 \lambda^3 - 7*\lambda^2 + 14 *\lambda - 8 = 0$
		\item solve the characteristic equation to obtain the three eigenvalues (roots of the cubic polynomial):
				\begin{itemize}
					\item $(\lambda -4)(\lambda - 1)(\lambda -2) = 0$
					\item$\lambda_1 = 4 \qquad \lambda_2 = 2 \qquad \lambda_3 = 1$
				\end{itemize}
		\item determine the eigenvectors (one for each eigenvalue); assume one component of $\{p\} = 1$ and determine
		 	remaining two components.  If this assumption results in a vectors that does not satisfy:   \\
			$\Big[ [T] - \lambda [I] \Big] \{p\} = \{0\}$; then assume a value of 0 for the component of $\{p\}$.
			 \begin{equation*} \Big[ [T] - \lambda [I] \Big] \{p\} = \{0\} \Rightarrow
			 \begin{bmatrix} 
				3 - \lambda_i & 0 &-1 \\ 0 & 1- \lambda_i & 0 \\ -1 & 0 & 3 -\lambda_i \\

			\end{bmatrix}  \begin{Bmatrix} p_{i,1} \\ p_{i,2} \\ p_{i,3} \end{Bmatrix} = 
			\begin{Bmatrix} 0 \\ 0 \\ 0 \end{Bmatrix} \end{equation*}
			\begin{itemize}
     				\item for $\lambda_1 = 4: \langle p_1 \rangle = \langle1, 0, -1 \rangle$; Normalize $\Rightarrow 
					\langle \frac{1}{\sqrt{2}}, 0, -\frac{1}{\sqrt{2}} \rangle$
  				
				\item for $\lambda_2 = 2: \langle p_2 \rangle = \langle1, 0, 1 \rangle$; Normalize $ \Rightarrow 
					\langle \frac{1}{\sqrt{2}}, 0, \frac{1}{\sqrt{2}} \rangle$
				
				\item for $\lambda_3 = 1:  \langle p_3 \rangle=\langle p_1\rangle \times 
					\langle p_2 \rangle = \langle 0, -1, 0 \rangle$; 
					only two of the vectors are independent, the third can be calculated from 
					the two independent vectors to form an orthonormal set of vectors.
			\end{itemize}
			\item the principal basis:  $\bm{p_a} $ in terms of $\bm{e_i}: \qquad \bm{p_a} \Rightarrow \begin{matrix}
				\bm{p_1} = \frac{1}{\sqrt{2}} \bm{e_1}  - \frac{1}{\sqrt{2}}  \bm{e_3} \\
				\bm{p_2} = \frac{1}{\sqrt{2}} \bm{e_1}  + \frac{1}{\sqrt{2}}  \bm{e_3} \\
				\bm{p_3} = -1 \bm{e_2}  \\
				\end{matrix}$
	\end{itemize}

% ==== problem 4 ====
\section{Find the components of $\bm{T}, \bm{T^2}$ and $\bm{T^3}$ in the $\bm{p_a}$ basis and calculate respective 
		invariants:}
	\begin{itemize}
		\item $\overset{p-e}{[a]} = \begin{bmatrix} 
			\frac{1}{\sqrt{2}} & 0 & -\frac{1}{\sqrt{2}} \\
			\frac{1}{\sqrt{2}} & 0 & \frac{1}{\sqrt{2}} \\
			0 & -1 & 0
			\end{bmatrix}$
		\item $\bm{T} \Rightarrow \overset{p-p}{[T]} = \overset{p-e}{[a]} \overset{e-e}{[T]} \overset{e-p}{[a]}
			= \begin{bmatrix}	4 & 0 & 0 \\
							0 & 2 & 0 \\
							0 & 0 & 1
						\end{bmatrix}$
		\item $\bm{T^2} \Rightarrow \overset{p-p}{[T]} \overset{p-p}{[T]} = \begin{bmatrix}
			16 & 0 & 0 \\
			0 & 4 & 0 \\
			0 & 0 & 1
			\end{bmatrix}$

		\item $\bm{T^3} \Rightarrow \overset{p-p}{[T]}\overset{p-p}{[T]}\overset{p-p}{[T]}= \begin{bmatrix}
			64 & 0 & 0 \\
			0 & 8 & 0 \\
			0 & 0 & 1
			\end{bmatrix}$	
		\item $I_T = 4+2+1 = 7$
		\item$II_T = 16 + 4 + 1 = 21$
		\item$III_T = 64 + 8 + 1 = 73 $ ... same as before, just like they should be.
	\end{itemize}

% ==== problem 5 ====
\section{Set up the transformation matrix between $\bm{p_a}$ and $\bm{e_i}$}
	\begin{itemize}	
		\item $\overset{e-p}{[a]} = tr \left( \overset{p-e}{[a]} \right)= \begin{bmatrix} 
			\frac{1}{\sqrt{2}} & \frac{1}{\sqrt{2}} & 0 \\
			0& 0 & -1 \\
			-\frac{1}{\sqrt{2}} & \frac{1}{\sqrt{2}} & 0
			\end{bmatrix}$
			
		\item $\bm{T} \Rightarrow \overset{e-e}{[T]} = \overset{e-p}{[a]} \overset{p-p}{[T]} \overset{p-e}{[a]}
			= \begin{bmatrix}	3 & 0 & -1 \\
							0 & 1 & 0 \\
							-1 & 0 & 3
						\end{bmatrix}$			
	\end{itemize}
	
% ==== problem 6 ====
\section{}
	\subsection {Obtain the values of the invariants in the principal basis $\hat{I_T}, \hat{II_T}, \hat{III_T}$}
		\begin{itemize}
		\item $\hat{I_T}  = tr\left( \overset{p-p}{[T]} \right) = 7$
		\item $\hat{II_T} = \frac{1}{2}(II_T - I^2_T) = -14 $
		\item $\hat{III_T} = \frac{1}{6}(I^3_T - 3I_T II_T + 2III_T) = det\left(\overset{p-p}{[T]} \right) = 8$
		\end{itemize}
	\subsection{Show that the Cayley-Hamilton theorem holds using components in the $\bm{e_i}$ system.}
		\begin{itemize}
		\item $\bm{T^2} \Rightarrow \overset{e-e}{[T]} \overset{e-e}{[T]} = \begin{bmatrix}
			10 & 0 & -6 \\
			0 & 1 & 0 \\
			-6 & 0 & 10
			\end{bmatrix}$

		\item $\bm{T^3} \Rightarrow \overset{e-e}{[T]}\overset{e-e}{[T]}\overset{e-e}{[T]}= \begin{bmatrix}
			36 & 0 & -28 \\
			0 & 1 & 0 \\
			-28 & 0 & 36
			\end{bmatrix}$	
		\item characteristic invariants:  $I^*_T = 7, II^*_T = -14, III^*_T = 8$
		\item Cayley-Hamilton Theorem for the $\bm{e_i} $system, or in any system (which is why it may be written in 
			direct notation):\\ 
			$\bm{T^3} - I^*_T \bm{ T^2} - II^*_T -III^*_T \bm{I} =
			 \begin{bmatrix} 0 & 0 & 0 \\ 0 & 0 & 0 \\ 0 & 0 & 0\\ \end{bmatrix}$
		\end{itemize}
	\subsection{ With the use of components in either system, show that $\hat{III_T} = det(\bm{T})$}
		\begin{itemize}
		\item $\frac{1}{6} \Big[ I^3_T - 3I_T II_T + 2 III_T \Big] = \frac{1}{6} \Big[ 7^3 - 3 * 7 * 21 + 2 * 73 \Big] = 8$
		\item $ det \left( [T] \right) = 8$
		\end{itemize}
	
% ==== problem 7 ====
\section{Find the components of the tensor $\bm{T^{1/2}}$ in the $\bm{e_i}$ system, i.e., find $T^{1/2}_{ij} T^{1/2}_{jk}$:}
	\begin{itemize}
	\item from the spectral theorem, when a tensor of eigenvectors and a diagonal of eigenvalues are formed:
		\begin{itemize} 
			\item$\{\bm{p_1},\bm{ p_2}, \bm{p_3}\} \Rightarrow \overset{e-e}{[P]} \Rightarrow P_{ij} =\begin{bmatrix} 
					\frac{1}{\sqrt{2}} & \frac{1}{\sqrt{2}} & 0 \\
					0& 0 & -1 \\
					-\frac{1}{\sqrt{2}} & \frac{1}{\sqrt{2}} & 0
				\end{bmatrix} $
			 \item $\{\lambda_1, \lambda_2, \lambda_3\} [I] = [\Lambda] = \begin{bmatrix} 4 & 0 & 0 \\
			 															0 & 2 & 0 \\
																		0 & 0 & 1
																	\end{bmatrix}$
		\end{itemize}
	\item $T_{ik} = T_{ij}^{\frac{1}{2}} T_{jk}^{\frac{1}{2}} \Rightarrow \overset{e-e}{[P]^{-1}} [\Lambda] \overset{e-e}{[P]} = 
			\begin{bmatrix} 
				\frac{1}{\sqrt{2}} & \frac{1}{\sqrt{2}} & 0 \\
				0& 0 & -1 \\
				-\frac{1}{\sqrt{2}} & \frac{1}{\sqrt{2}} & 0
			\end{bmatrix} 
			\begin{bmatrix}
				 4 & 0 & 0 \\
				0 & 2 & 0 \\
				0 & 0 & 1
			\end{bmatrix}		
			\begin{bmatrix} 
				\frac{1}{\sqrt{2}} & 0 & -\frac{1}{\sqrt{2}} \\
				\frac{1}{\sqrt{2}} & 0 & \frac{1}{\sqrt{2}} \\
				0 & -1 & 0
			\end{bmatrix}$
	\end{itemize}
	
% ==== problem 8 ====
\section{Find the components of $\bm{T^{-1}}$ in the $\bm{e_i}$ system:}
	\begin{itemize}
		\item $\bm{T^{-1}} = \left( \bm{T^2} - \hat{I_T} \bm{T} - \hat{II_T}  \bm{I} \right) / \left( \hat{III_T} \right) \Rightarrow
			\frac{[T]^{cf}}{det\left( [T] \right)}$
		\item $\bm{T^{-1}} \Rightarrow \left(
			\begin{bmatrix}
				10 & 0 & -6 \\
				0 & 1 & 0 \\
				-6 & 0 & 10
			\end{bmatrix}
			- 7 \begin{bmatrix} 
				3 & 0 &-1 \\ 
				0 & 1 & 0 \\ 
				-1 & 0 & 3\\
			\end{bmatrix} 
			+14 \begin{bmatrix}
				 1 & 0 & 0 \\
				0 & 1 & 0 \\
				0 & 0 & 1
			\end{bmatrix}	\right) / 8 = 
			\begin{bmatrix}
				3/8 & 0 & 1/8 \\
				0 & 1 & 0 \\
				1/8& 0& 3/8
			\end{bmatrix}$
		\item $\overset{p-p}{[T]^{-1}}  = \overset{p-e}{[a]} \overset{e-e}{[T]^{-1}} \overset{e-p}{[a]} =
			\begin{bmatrix}
				0.25 & 0 & 0 \\
				0 & 0.5 & 0 \\
				0 & 0 & 1
			\end{bmatrix}$
	\end{itemize}
%--------------------------------------------------------------
%     You don't have to mess with anything below this line.
% --------------------------------------------------------------
 
\end{document}