% --------------------------------------------------------------
% This is all preamble stuff that you don't have to worry about.
% Head down to where it says "Start here"
% --------------------------------------------------------------
 
\documentclass[10pt, letterpaper]{article}
 
\usepackage[margin=1in]{geometry} 
\usepackage{mathtools, amsthm, amssymb, changepage, enumitem}
\usepackage[english]{babel}
\usepackage{bm}
\usepackage{tensor}
\usepackage{xfrac}
\usepackage{gensymb}

\renewcommand\thesection{ \arabic{section}}
\renewcommand\thesubsection{(\alph{subsection})}
\renewcommand\thesubsubsection{(\roman{subsubsection})}
 
\begin{document}
 
% --------------------------------------------------------------
%                         Start here
% --------------------------------------------------------------
 
\title{ASSIGNMENT 3}%replace X with the appropriate number
\author{Brandon Lampe\\ %replace with your name
ME 512 - Continuum Mechanics} %if necessary, replace with your course title
 
\maketitle

%=====================================
% PROBLEM 1
%===================================== 
\section{Suppose the following:}  
	\begin{equation*} 
		T_{pq}  \Rightarrow	\begin{bmatrix} -1 & 2 & 3 \\ 2 & -2 & 2 \\ 4 & 3 & 4\\
						 \end{bmatrix} \qquad		 
		u_i \Rightarrow (1, -2, 2) \qquad
		v_i \Rightarrow (-2, 1, -3)
	\end{equation*}
	
	Notes on the nomenclature:
	\begin{itemize}
		\item $\bm{I}$ denotes the identity tensor.
		\item Upper case letters indicate a second order tensor and lower case letters indicate a vector.
		\item An implied basis of $\bm{e_i}$ and $\bm{e_i} \otimes \bm{e_j}$ as used for vectors and 			tensors, respectively.
	\end{itemize}
	\begin{equation*} 
		\bm{T}  =	\begin{matrix*}
			 -1( \bm{e_1} \otimes \bm{e_1}) +  
			 2 (\bm{e_1} \otimes \bm{e_2})+  
			 3 (\bm{e_1} \otimes \bm{e_3}) \\
			 + 2 (\bm{e_2} \otimes \bm{e_1}) 
			 -2 (\bm{e_2} \otimes \bm{e_2})+ 
			 2 (\bm{e_2} \otimes \bm{e_3})  \\
			 +4 (\bm{e_3} \otimes \bm{e_1}) +
			 3 (\bm{e_3} \otimes \bm{e_2})+  
			 4 (\bm{e_3} \otimes \bm{e_3}) \\
		 \end{matrix*} \qquad		 
		\bm{u} = 1 \bm{e_1} -2 \bm{e_2} +2 \bm{e_3} \qquad
		\bm{v} = -2 \bm{e_1} + 1\bm{e_2} - 3\bm{e_3}
	\end{equation*}
	\begin{itemize}
		\item or more succinctly as:\\
	$\bm{T} = T_{pq} (\bm{e_p} \otimes \bm{e_q}) \qquad \bm{u} = u_i\bm{e_i} \qquad \bm{v} = v_i \bm{e_i}$
	\end{itemize}
%------------ a --------------
	\subsection{? = $\bm{u} \bullet \bm{v}$}
		\begin{enumerate}[label = (\roman*)]
			\item $w = \bm{u} \bullet \bm{v}$ (a scalar)
			\item $w = \bm{u} \bullet \bm{v}$\\
				$w = u_i v_i = u_1 v_1 + u_2 v_2 + u_3 v_3 $\\
				$w = <u> \{v\}$
			\item $w = -10$
		\end{enumerate}
		
%------------ b --------------
	\subsection{? = $\bm{T} \bullet \bm{u}$}
		\begin{enumerate}[label = (\roman*)]
			\item $w_i \bm{e_i} \Rightarrow \bm{T} \bullet \bm{u} \Rightarrow
				T_{ij}(\bm{e_i} \otimes \bm{e_j}) \bullet v_k \bm{e_k}
							= T_{ij}v_k \delta_{jk} \bm{e_i} = T_{ij}v_j \bm{e_i}$ (components and base vector)
			\item $\bm{w} = \bm{T} \bullet \bm{u}$ \\
				 $w_i = T_{ij} u_j$\\				 			
				$\{w\} = [T] \{u\}$
			\item $\bm{w} = 1 \bm{e_1} + 10 \bm{e_2} + 6 \bm{e_3}$
		\end{enumerate}		

%------------ c --------------
	\subsection{? = $\bm{u} \bullet \bm{T^T}$}
		\begin{enumerate}[label = (\roman*)]
			\item $w_i \bm{e_i} \Rightarrow \bm{u} \bullet \bm{T^T} = \bm{T} \bullet \bm{u}$ (components and base vector)
			\item $\bm{w} = \bm{T} \bullet \bm{u}$ \\
				 $w_i = T_{ij} u_j$\\				 			
				$\{w\} = [T] \{u\}$
			\item $\bm{w} = 1 \bm{e_1} + 10 \bm{e_2} + 6 \bm{e_3}$
		\end{enumerate}		

%------------ d --------------
	\subsection{? = $\bm{v} \bullet \bm{T} \bullet \bm{u}$}
		\begin{enumerate}[label = (\roman*)]
			\item $w = \bm{v} \bullet \bm{T} \bullet \bm{u} = \bm{T} \cdot \cdot (\bm{u} \otimes \bm{v})$ 
				(dot product with two vectors is a scalar)
			\item $w = \bm{v} \bullet \bm{T} \bullet \bm{u}$ \\
				 $w = T_{pq} v_p u_q$\\				 			
				$w = \langle v \rangle [T] \{u\}$
			\item $w = -10$
		\end{enumerate}
		
%------------ e --------------
	\subsection{? = $\bm{u} \otimes \bm{v}$}
		\begin{enumerate}[label = (\roman*)]
			\item $? \Rightarrow$ the "tensor product" or "dyadic multiplication" between two base vectors 
					operates on an arbitrary vector 
					e.g., $(\bm{u} \otimes \bm{v}) \bullet \bm{z} = \bm{u} (\bm{v} \bullet \bm{z})$.  
			\item	$\bm{u} \otimes \bm{v}$
				$\bm{u_i} \otimes \bm{v_i} \\
				\{\bm{u} \} \otimes \langle \bm{v} \rangle$
			\item $\begin{bmatrix*}
			 (\bm{u_1} \otimes \bm{v_1}) &  
			 (\bm{u_1} \otimes \bm{v_2}) & 
			 (\bm{u_1} \otimes \bm{v_3}) \\
			 (\bm{u_2} \otimes \bm{v_1}) &
			 (\bm{u_2} \otimes \bm{v_2}) &
			 (\bm{u_2} \otimes \bm{v_3})  \\
			 (\bm{u_3} \otimes \bm{v_1}) &
			 (\bm{u_3} \otimes \bm{v_2}) & 
			 (\bm{u_3} \otimes \bm{v_3}) \\
		 \end{bmatrix*} $
				
		\end{enumerate}	

%------------ f --------------
	\subsection{? = $\bm{I} \cdot \cdot \bm{T}$}
		\begin{enumerate}[label = (\roman*)]
			\item $w = \bm{I} \cdot \cdot \bm{T}$ (The inner product between two tensors results in a scalar)
			\item $w = \bm{I} \cdot \cdot \bm{T} = \bm{T^{tr}}$\\
				 $w = T_{pp}$\\				 			
				$w = tr( [T] )$
			\item $w = 1$
		\end{enumerate}
			
%------------ iv --------------
	\subsection{Determine the components of $\bm{T^{sym}}$ and $\bm{T^{skw}}$ of $\bm{T}$.\\
				Part (iv) of problem 1}
		\begin{itemize}
			\item The symmetric part of $\bm{T}$:\\
				$\bm{T^{sym}} = \frac{1}{2} \begin{bmatrix} \bm{[T]}  + \bm{[T]^T} \end{bmatrix} \Rightarrow
				\begin{bmatrix} -1 & 2 & 7/2\\
							2 & -2 & 5/2 \\
							7/2 & 5/2 & 4
				\end{bmatrix}$
			
			\item The skew-symmetric part of $\bm{T}$:\\
				$\bm{T^{skw}} = \frac{1}{2} \begin{bmatrix} \bm{[T]}  - \bm{[T]^T} \end{bmatrix} \Rightarrow
				\begin{bmatrix} 0 & 0 & -1/2\\
							0 & 0 & -1/2 \\
							1/2 & 1/2 & 0
				\end{bmatrix}$

		\end{itemize}

%------------ v --------------
	\subsection{Determine the components $b_i, c_i,$ and $d_i$.\\
				Part (v) of problem 1}
			\begin{itemize}
			% ---- bi ----
				\item $b_i = \frac{1}{2} \mathcal{E}_{ijk} T_{jk} \qquad$ or $ \qquad
					\bm{b} = \frac{1}{2} \bm{{\overset{3}{\mathcal{E}}}} \cdot \cdot \bm{T}$ \\
					$b_i \Rightarrow 
						 \begin{matrix} 
							b_1 = \frac{1}{2}(T_{23} - T_{32}) & = & -1/2\\
							b_2 = \frac{1}{2}(T_{13} - T_{31}) & = & -1/2\\
							b_3 = \frac{1}{2}(T_{12} - T_{21}) & = & 0\\
						\end{matrix}$
			% ---- ci ----
				\item $c_i = \frac{1}{2} \mathcal{E}_{ijk} T_{jk}^{sym} \qquad$ or $ \qquad
					\bm{c} = \frac{1}{2} \bm{{\overset{3}{\mathcal{E}}}} \cdot \cdot \bm{T^{sym}}$ \\
					$c_i \Rightarrow 
						 \begin{matrix} 
							c_1 = \frac{1}{2}(T_{23}^{sym} - T_{32}^{sym}) & = & 0\\
							c_2 = \frac{1}{2}(T_{13}^{sym} - T_{31}^{sym}) & = & 0\\
							c_3 = \frac{1}{2}(T_{12} ^{sym}- T_{21}^{sym}) & = & 0\\
						\end{matrix}$

			% ---- di ----
				\item $d_i = \frac{1}{2} \mathcal{E}_{ijk} T_{jk}^{skw} \qquad$ or $ \qquad
					\bm{d} = \frac{1}{2} \bm{{\overset{3}{\mathcal{E}}}} \cdot \cdot \bm{T^{skw}}$ \\
					$d_i \Rightarrow 
						 \begin{matrix} 
							d_1 = \frac{1}{2}(T_{23}^{skw} - T_{32}^{skw}) & = & 1/2\\
							d_2 = \frac{1}{2}(T_{13}^{skw} - T_{31}^{skw}) & = & 1/2\\
							d_3 = \frac{1}{2}(T_{12} ^{skw}- T_{21}^{skw}) & = & 0\\
						\end{matrix}$
			\end{itemize}

% ======================================================================
\section{Show that $\bm{v} = (\bm{v} \cdot \bm{n}) \bm{n} + \bm{n} \times (\bm{v} \times \bm{n})$ holds for all $\bm{n}$ and that 		this represents a resolution (or projection) of $\bm{v}$ into vectors parallel and perpendicular to $\bm{n}$, where
		 $\bm{n}$ is a unit vector:}
		 
	\begin{enumerate}[label = (\roman*)]
		\item set the right term  $\bm{n} \times (\bm{v} \times \bm{n}) = \bm{u} \times (\bm{v} \times \bm{w}) $ and solve;
		\item	$\bm{v} \times \bm{w} \Rightarrow v_i \bm{e_i} \times w_i \bm{e_i} = v_i w_i \mathcal{E}_{ijk} \bm{e_k} $;
		\item$ \bm{u} \times (\bm{v} \times \bm{w}) \Rightarrow u_l \bm{e_l} \times v_i w_i \mathcal{E}_{ijk} \bm{e_k}
			= u_l v_i w_j  \mathcal{E}_{ijk} \mathcal{E}_{klm} \bm{e_m} $;
		\item replace $\mathcal{E}_{ijk} \mathcal{E}_{klm} $ using the $\mathcal{E} - \delta$ identity:\\
			free indices: $\mathit{i, j, l, m}$; therefore:  $\mathcal{E}_{ijk} \mathcal{E}_{klm} = 
			\delta_{il} \delta_{jm} - \delta_{im} \delta_{jl}$;
		\item $(\delta_{il} \delta_{jm} - \delta_{im} \delta_{jl}) u_l v_i w_j \bm{e_m}$;
		\item remove Kronecker Deltas that contain a repeated (dummy) index:  $u_i v_i w_m \bm{e_m} - u_j w_j v_m 
			\bm{e_m}$;
		\item replace with original values and write in direct notation: $(\bm{n} \cdot \bm{v}) \bm{n} - 
			(\bm{n} \cdot \bm{n}) \bm{v} = 0$
		\item where in (vii) $\bm{n} \cdot \bm{v} = v^n$, then $v^n \bm{n} = \bm{v}$, and $\bm{n} \cdot \bm{n} = 1$
		\item the left term $(\bm{v} \cdot \bm{n}) \bm{n}= \Arrowvert \bm{v} \Arrowvert \bm{n} = \bm{v}$
		\item therefore $\bm{v} = (\bm{v} \cdot \bm{n}) \bm{n} + \bm{n} \times (\bm{v} \times \bm{n})$
		\item Additionally, the resolution of $\bm{v}$ parallel to $\bm{n}$ is: $(\bm{v} \cdot \bm{n}) \bm{n}$ as shown in (ix)\\
			the resolution of $\bm{v}$ perpendicular to $\bm{n}$ is shown in (vii): $\bm{n} \times (\bm{v} \times \bm{n})
			 =(\bm{n} \cdot \bm{v}) \bm{n} - 
			(\bm{n} \cdot \bm{n}) \bm{v} = \Arrowvert v \Arrowvert \bm{n} - \bm{v}$
					
	\end{enumerate}
			
% == problem 3 ====================================================================
\section{Suppose \textbf{T} and \textbf{U} are second-order tensors.}
	\subsection{Show $tr(\bm{T} \cdot \bm{U}) = tr(\bm{T^T} \cdot \bm{U})$ if either $\bm{T}$ or $\bm{U}$ are symmetric.}
		\begin{itemize}
			\item if $\bm{T}$ is symmetric then: $\bm{T} \Rightarrow T_{ij} = T_{ij}^T = T_{ij}$
			\item define $\bm{A}$ such that $A_{ik} = T_{ij} U_{jk} = T_{ij}^T U_{jk}$
			\item therefore $tr(\bm{A}) = tr(\bm{T} \cdot \bm{U}) = tr(\bm{T^T} \cdot \bm{U})$
		\end{itemize}
% ---- b ----
	\subsection{Show $tr(\bm{T} \cdot \bm{U}) = 0$ if one of the tensors is skew-symmetric and the other is symmetric.}
		\begin{itemize}
			\item symmetric tensor $\Rightarrow T_{ij} = T_{ji}$
			\item skew-symmetric tensor $\Rightarrow U_{ij} = -U_{ji}$
			\item $tr(\bm{T} \cdot \bm{U}) \Rightarrow T_{ij} U_{ij}  =- T_{ij} U_{ji} = -T_{ji} U_{ji}$; this can only be true if
				$tr(\bm{T} \cdot \bm{U}) = 0$
		\end{itemize}
% == PROBLEM  =============================
\section{}
	Handwritten on paper and did not complete, see final page of assignment.
% == PROBLEM 5 =============================
\section{Given:}
	\begin{equation*}
		\begin{matrix}			& \bm{E_1} 	& \bm{E_1} 	& \bm{E_1} \\
					\bm{e_1}	& \pi / 2		& \pi / 4		& 3 \pi / 4	\\
					\bm{e_2}	& \pi / 4		& \pi / 3		& \pi / 3 \\
					\bm{e_3}	& \pi / 4		& 2 \pi / 3		& 2 \pi / 3\\ 
					\end{matrix} \qquad 
				\overset{e}{\{v\}} = \begin{Bmatrix} 1 \\ -2 \\ 3 \end{Bmatrix} \qquad
				\overset{e-e}{[T]} = \begin{bmatrix}
					 2 & 0 & -3 \\
					0 & 6 & 0 \\
					-3 & 0 & 4 \end{bmatrix}	\end{equation*}
% ---- a ------    
    	\subsection{}
		\begin{itemize}			
			\item $ \bm{e_i}$ in terms of $\bm{E_A}: \qquad \bm{e_i} \Rightarrow \begin{matrix} 
				\bm{e_1}  = \\ \bm{e_2} = \\ \bm{e_3} = \\ 
				\end{matrix} \begin{matrix}
				\cos (\pi/2) \bm{E_1} + \cos(\pi/4) \bm{E_2} + \cos(3 \pi / 4) \bm{E_3} \\
				\cos (\pi/4) \bm{E_1} + \cos(\pi/3) \bm{E_2} + \cos(\pi /3) \bm{E_3} \\
				\cos (\pi/4) \bm{E_1} + \cos(2\pi/3) \bm{E_2} + \cos(2 \pi /3) \bm{E_3} \\
					\end{matrix}$ 
					
			\item $ \bm{E_A}$ in terms of $\bm{e_i}: \qquad \bm{E_A} \Rightarrow \begin{matrix} 
				\bm{E_1}  = \\ \bm{E_2} = \\ \bm{E_3} = \\ 
				\end{matrix} \begin{matrix}
				\cos (\pi/2) \bm{e_1} + \cos(\pi/4) \bm{e_2} + \cos( \pi / 4) \bm{e_3} \\
				\cos (\pi/4) \bm{e_1} + \cos(\pi/3) \bm{e_2} + \cos(2\pi /3) \bm{e_3} \\
				\cos (3\pi/4) \bm{e_1} + \cos(\pi/3) \bm{e_2} + \cos(2 \pi /3) \bm{e_3} \\
					\end{matrix}$ 
					
			\item Verify that $\bm{E_A}$ is a right-handed orthonormal system: 
				\begin{itemize}
					\item Right-handed and orthogonal because:
						$\begin{matrix}
						 \bm{E_1} \times \bm{E_2} = \bm{E_3} \\
						 \bm{E_2} \times \bm{E_3} = \bm{E_1} \\
						 \bm{E_3} \times \bm{E_1} = \bm{E_2} \\
						 \end{matrix}$
				
					\item Normal (i.e., unit length of one) because:
						$\begin{matrix}
						 \bm{E_1} \cdot \bm{E_1} = 1 \\
						 \bm{E_2} \cdot \bm{E_2} = 1 \\
						 \bm{E_3} \cdot \bm{E_3} = 1 \\
						 \end{matrix}$
				\end{itemize}
		\end{itemize}
% ---- b -----			
	\subsection{Obtain the transformation matrix:}
		\begin{itemize}
			\item $\overset{e -E}{[a]} = \begin{bmatrix}
						0	&	\sfrac{\sqrt{2}}{2}	&	-\sfrac{\sqrt{2}}{2} \\
						\sfrac{\sqrt{2}}{2}	&	\sfrac{1}{2}		& 	\sfrac{1}{2} \\
						\sfrac{\sqrt{2}}{2}	&	-\sfrac{1}{2}	&	-\sfrac{1}{2}
						\end{bmatrix}$
						
			\item $\overset{e- E}{[a]^T} =\overset{E- e}{[a]}= \begin{bmatrix}
						0	&	\sfrac{\sqrt{2}}{2}	&	\sfrac{\sqrt{2}}{2} \\
						\sfrac{\sqrt{2}}{2}	&	\sfrac{1}{2}		& 	-\sfrac{1}{2} \\
						-\sfrac{\sqrt{2}}{2}	&	\sfrac{1}{2}	&	-\sfrac{1}{2}
						\end{bmatrix}$
			
			\item because the transformation matrix is orthonormal: $\overset{e -E}{[a]} \overset{E- e}{[a]}= [I]$
		\end{itemize}
% ---- c ----
	\subsection{Find the components of $\bm{v}$ in the $\bm{E_A}$ system:}
		\begin{itemize}
			\item $\overset{E}{\{v\}} =  \overset{E- e}{[a]} \overset{e}{\{v\}} = \begin{Bmatrix}
				\sfrac{\sqrt{2}}{2} \\ \sfrac{(\sqrt{2} - 5)}{2} \\ \sfrac{(-\sqrt{2}-5)}{2}	
				\end{Bmatrix}$
				
			\item $\overset{e}{\{v\}} = \overset{e -E}{[a]}\overset{E}{\{v\}} = \begin{Bmatrix}
				1 \\ -2 \\ 3	
				\end{Bmatrix}$
		\end{itemize}
		
% ---- d ----
	\subsection{Find the components of $\bm{T}$ in the $\bm{E_A}$ system:}
		\begin{itemize}
			\item $\overset{E-E}{[T]} =  \overset{E -e}{[a]} \overset{e-e}{[T]}  \overset{e -E}{[a]} =
				\begin{bmatrix}	5.00 & -0.79 & 2.21 \\
							-0.79 & 5.62 & 1.50 \\
							2.21 & 1.50 & 1.38 
					\end{bmatrix} $
		\end{itemize}
		
% ---- e ----
	\subsection{Find the mixed components of $\overset{E-e}{[T]}$ and $\overset{e-E}{[T]}$ :}
		\begin{itemize}
			\item $\overset{E-e}{[T]} =  \overset{E -e}{[a]} \overset{e-e}{[T]} = \overset{E-E}{[T]} \overset{E -e}{[a]} =
					\overset{e-E}{[T]^T} =
				\begin{bmatrix}	-2.12 & 4.24 & 2.82 \\
							2.91 & 3.00 & -4.12 \\
							0.09 & 3.00 & 0.12 
					\end{bmatrix} $
			\item $\overset{e-E}{[T]} =  \overset{e -E}{[a]} \overset{E-E}{[T]} = \overset{e-e}{[T]} \overset{e -E}{[a]} =
					\overset{E-e}{[T]^T} =
				\begin{bmatrix}	-2.12 & 2.91 & 0.09 \\
							4.24 & 3.00 & 3.00 \\
							2.82 & -4.12 & 0.12 
					\end{bmatrix} $
		\end{itemize}
		
% ==== Problem 6 ====
\section{Obtain the transformation matrices for the transforming components from:}
	\begin{itemize}
		\item assuming each basis is orthonormal
	\end{itemize}

% --- a ---
	\subsection{ the $\bm{e_i}$ basis to the $\bm{E_A}$ basis:}
		\begin{itemize} 
			\item $\bm{E_A} \Rightarrow \begin{matrix*} 
				\bm{E_1} & = & \cos(\alpha) \bm{e_1}  + 
				 \cos(90 \degree - \alpha) \bm{e_2} + \cos(90 \degree)\bm{e_3} \\
				 
				 \bm{E_2} & = & \cos(\alpha + 90 \degree) \bm{e_1}  + 
				 \cos(90 \degree) \bm{e_2} + \cos(90 \degree)\bm{e_3} \\
				 
				 \bm{E_3} & = & \cos(90 \degree) \bm{e_1} + 
				 \cos(90 \degree ) \bm{e_2} + \cos(0)\bm{e_3} \\
			\end{matrix*}$
		
		\item $\overset{E - e}{[a]} = \begin{bmatrix}
				\cos(\alpha) & \sin( \alpha)  & 0 \\
				-\sin(\alpha) & \cos(\alpha ) &0 \\
				0 & 0 & 1 \\
			\end{bmatrix}$

% ---- b ----		
			\end{itemize}
	\subsection{ the $\bm{e_i}$ basis to the $\bm{E_A}$ basis:}
		\begin{itemize} 
			\item $\bm{E_A} \Rightarrow \begin{matrix*} 
				\bm{E_1} & = & \cos(0) \bm{g_1}  + 
				 \cos(90 \degree) \bm{g_2} + \cos(90 \degree)\bm{g_3} \\
				 
				 \bm{E_2} & = & \cos(90 \degree) \bm{g_1}  + 
				 \cos(-\beta) \bm{g_2} + \cos(-\beta -90 \degree)\bm{g_3} \\
				 
				 \bm{E_3} & = & \cos(90 \degree) \bm{g_1} + 
				 \cos(\beta - 90 \degree) \bm{g_2} + \cos(90\degree + \beta)\bm{g_3} \\
			\end{matrix*}$
		
		\item $\overset{E - g}{[a]} = \begin{bmatrix}
				1 & 0  & 0 \\
				0 & \cos(\beta ) & -\sin(\beta) \\
				0 & \sin(\beta) & \cos(\beta) \\
			\end{bmatrix}$
		
			\end{itemize}
	
	\subsection{ the $\bm{e_i}$ basis to the $\bm{E_A}$ basis:}
		\begin{itemize}
			\item $\overset{e - g}{[a]} = \overset{e - E}{[a]} \overset{E - g}{[a]} = \overset{E- e}{[a]^T} \overset{E - g}{[a]} =
				\begin{bmatrix}		\cos \alpha & -\sin\alpha \cos \beta & \sin \alpha \cos \beta \\
								\sin \alpha & \cos \alpha \cos \beta & - \cos \alpha \sin \beta \\
								0 & \sin \beta & \cos \beta \\
							\end{bmatrix}$
			\item to check that this transformation is still orthonormal:  $\overset{e - g}{[a]} \overset{g - e}{[a]} = [I]$
		\end{itemize}
% --------------------------------------------------------------
%     You don't have to mess with anything below this line.
% --------------------------------------------------------------
 
\end{document}