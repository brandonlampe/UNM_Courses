% --------------------------------------------------------------
% This is all preamble stuff that you don't have to worry about.
% Head down to where it says "Start here"
% --------------------------------------------------------------
 
\documentclass[10pt, letterpaper]{article}
 
\usepackage[margin=1in]{geometry} 
\usepackage{mathtools, amsthm, amssymb, changepage, enumitem}
 

\renewcommand\thesection{ \arabic{section}}
\renewcommand\thesubsection{\thesection.\alph{subsection}}
\renewcommand\thesubsubsection{(\roman{subsubsection})}
 
\begin{document}
 
% --------------------------------------------------------------
%                         Start here
% --------------------------------------------------------------
 
\title{Bonus 2}%replace X with the appropriate number
\author{Brandon Lampe\\ %replace with your name
ME 571 - Material Science} %if necessary, replace with your course title
 
\maketitle
 
\section{Define and provide examples of isotropic and homogenous materials:}  
	\subsection{isotropic materials:}
		\begin{itemize}
			\item definition:  a substance having material properties (e.g., Young's modulus and Poisson's ratio) that are uniform in all spatial directions i.e., they properties do not vary with direction;
			\item examples of isotropic materials:  steel, glass;
			\item examples of anisotropic materials:  carbon fiber, wood;
		\end{itemize}
	\subsection{homogenous materials:}
		\begin{itemize}
			\item definition:  a material having uniform composition and properties at every point in the material.  Homogeneity of a material is dependent on the scale at which the material is observed e.g., atomic, crystal, bulk scales;
			\item examples of materials with a homogenous crystal structure:  halite, gold  
			\item a uniform sandstone (e.g., quartz bound together with a calcite matrix) is typically not isotropic (i.e., properties vary with direction) but the bulk material is homogenous.
		\end{itemize}			 

% --------------------------------------------------------------
%     You don't have to mess with anything below this line.
% --------------------------------------------------------------
 
\end{document}