\documentclass[letterpaper, 10pt, oneside]{article}

\usepackage{tikz}
\usepackage[english]{babel}
\usepackage[margin=1in]{geometry}
\usepackage{mathtools}
\usepackage{amsmath}
\usepackage{graphicx}
\usepackage{framed} % for framed equations
\usepackage{empheq} % for framed equations
\usepackage{mathrsfs} % for script fonts
\usepackage[procnames]{listings} % for code listings
\usepackage{color} % for colored code listings

%%%%% new environment and new command definitions
\newenvironment{dd}[1]{
	\noindent
	\textbf{\normalsize{#1}}
	\hspace{0.1in}
	\small
	\rmfamily
	}
	{\medskip}
\newcommand{\be}{\begin{equation}}
\newcommand{\ee}{\end{equation}} 
\newcommand{\bes}{\begin{equation*}}
\newcommand{\ees}{\end{equation*}} 
\newcommand{\as}[1]{\begin{align*}#1\end{align*}}
\newcommand{\an}[1]{\begin{align}#1\end{align}}
\newcommand{\bdd}{\begin{dd}}
\newcommand{\edd}{\end{dd}} 
\newcommand{\bi}{\begin{itemize}}
\newcommand{\ei}{\end{itemize}}  
\newcommand{\boxedeq}[2]{\begin{empheq}[box={\fboxsep=6pt\fbox}]{align}\label{#1}#2\end{empheq}}
\newcommand{\coloredeq}[2]{\begin{empheq}[box=\colorbox{lightgreen}]{align}\label{#1}#2\end{empheq}}

%%%%% for code listings
\definecolor{keywords}{RGB}{255,0,90}
\definecolor{comments}{RGB}{0,0,113}
\definecolor{red}{RGB}{160,0,0}
\definecolor{green}{RGB}{0,150,0}
\lstset{language=Python, 
        basicstyle=\ttfamily\small, 
        keywordstyle=\color{keywords},
        commentstyle=\color{comments},
        stringstyle=\color{red},
        showstringspaces=false,
        identifierstyle=\color{green},
        procnamekeys={def,class}}

% Swap the definition of \abs* and \norm*, so that \abs
% and \norm resizes the size of the brackets, and the 
% starred version does not. 
\DeclarePairedDelimiter\abs{\lvert}{\rvert}%
\DeclarePairedDelimiter\norm{\lVert}{\rVert}%

%%%%%%%%%%%%%%%%%%%%%%%%%%%%%%%%%%%%%%%%%%%%%%%%%%%%%%%%%%%%%%%%%
% begin the document
%%%%%%%%%%%%%%%%%%%%%%%%%%%%%%%%%%%%%%%%%%%%%%%%%%%%%%%%%%%%%%%%%	
\title{CE 598\\ Practical Peridynamics\\ Assignment 8}
\author{Brandon Lampe}
% \date{October 3, 2015} % delete this line to display the current date
\begin{document}
\maketitle
%%%%%%%%%%%%%%%%%%%%%%%%%%%%%%%%%%%%%%%%%%%%%%%%%%%%%%%%%%%%%%%%%
% problem 1
%%%%%%%%%%%%%%%%%%%%%%%%%%%%%%%%%%%%%%%%%%%%%%%%%%%%%%%%%%%%%%%%%

\bf{Computer Problem}\\

Below is the code listing.  I do not think the results of this function are correct when the mesh is rotated, and I would like to discuss for clarification.\\

\begin{lstlisting}

function [stress strain] = computeStressStrain ...
            (xCur, Xref, L, LatRotAngle, bondList, E, nu, planeStressFlag)
numPtcls = size(xCur, 1);
numBonds = size(bondList, 2);
% oppBond = [2 1 4 3 6 5];

LatRotAngle = LatRotAngle*pi/180.;  %convert to radians

latRotZ = [cos(LatRotAngle) sin(LatRotAngle) 0;
           -sin(LatRotAngle)  cos(LatRotAngle) 0
           0 0 1];

%for plane stress only
D = (E/(1-nu^2))* ...
      [1    nu  0;
       nu   1   0;
       0    0   1];

% D = (E/((1+nu)*(1-2*nu)))* ...
%   [(1-nu)  nu   0          nu     0            0;
%    nu    (1-nu) 0          nu     0            0;
%    0      0     (1-2*nu)/2 0      0            0;
%    nu     nu    0          (1-nu) 0            0;
%    0      0     0          0      (1-2*nu)/2   0;
%    0      0     0          0      0           (1-2*nu)/2];
   
stress = zeros(numPtcls, 6);
strain = zeros(numPtcls, 6);
stretch_3 = zeros(numBonds/2,1);
stretch_6 = zeros(numBonds,1);
N = [   1           0               0;
        (0.5)^2     (sqrt(3)/2)^2   0.5*sqrt(3)/2;
        (-0.5)^2    (sqrt(3)/2)^2   -0.5*sqrt(3)/2];
    
N_inv = inv(N);

for iPtcl = 1:numPtcls
    for jBond = 1:numBonds
        if bondList(iPtcl,jBond)~= 0
            x_ptcl = xCur(iPtcl,1);
            y_ptcl = xCur(iPtcl,2);
            x_adj = xCur(bondList(iPtcl,jBond),1);
            y_adj = xCur(bondList(iPtcl,jBond),2);
            L_star = sqrt((x_ptcl - x_adj)^2 + (y_ptcl - y_adj)^2);
            stretch_6(jBond) = (L_star - L)/L;
        end
    end
    for jBond = 1:numBonds/2
        Bond = [1 3 5];
        coBond = [2 4 6];
        if stretch_6(Bond(jBond)) && stretch_6(coBond(jBond)) ~= 0
            stretch_3(jBond) = (stretch_6(Bond(jBond)) + ...
                stretch_6(coBond(jBond)))/2;
        elseif stretch_6(Bond(jBond)) || stretch_6(coBond(jBond)) == 0
            stretch_3(jBond) = stretch_6(Bond(jBond)) + ...
                stretch_6(coBond(jBond));
        end
    end

    strain(iPtcl,1:3) = latRotZ * N_inv * stretch_3;
    strain(iPtcl,4) = (strain(iPtcl,1) + strain(iPtcl,2))/2; % out of plane
    stress(iPtcl,1:3) = latRotZ * D * transpose(strain(iPtcl,1:3));
end

return
end
\end{lstlisting}
\end{document}